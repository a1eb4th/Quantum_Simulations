% Document title: ETSETB TFG LaTeX Template
% Version: 5.0
% Author: 2023 Orestes Mas Casals
% License: ETSETB TFG LaTeX Template by Orestes Mas is marked with CC0 1.0 Universal

%%%% TOOLS %%%%
\usepackage{lipsum}
\usepackage{etoolbox}   % <<< Provides some macros to compare text
\usepackage{iftex}      % <<< Detect TeX engine used to compile the document

%%%% MATH SETUP %%%%
\usepackage{mathtools}
\usepackage{amsmath}
\usepackage{amssymb}
\usepackage{braket}

%%%% INTERNATIONAL SYSTEM NOTATION FOR UNITS IN NUMBERS %%%%
\usepackage{siunitx}

%%%% FONTS SETUP %%%%
\iftutex   % True if compiling with Unicode engines (XeTeX/LuaTeX)
  \usepackage{fontspec}
  \usepackage[math-style=ISO,
    warnings-off={mathtools-colon, mathtools-overbracket},
  ]{unicode-math}
  %  This package is only a wrapper for the two packages libertinus-type1 (pdfLaTeX) and libertinus-otf (LuaLaTeX/XeLaTeX).
  %  The Libertinus fonts are similiar to Libertine and Biolinum, but come with math symbols. 
  \usepackage{libertinus}
  % Use DejaVu Mono for monospaced font
  \usepackage[
    mono=true,serif=false,sans=false,math=false,
    TT={Scale=0.80,FakeStretch=1.0}]{dejavu-otf}
\else    % True if compiling with 8-bit engines (LaTeX/pdfLaTeX)
  \usepackage[T1]{fontenc}
  %  This package is only a wrapper for the two packages libertinus-type1 (pdfLaTeX) and libertinus-otf (LuaLaTeX/XeLaTeX).
  %  The Libertinus fonts are similiar to Libertine and Biolinum, but come with math symbols. 
  \usepackage{libertinus,libertinust1math}
  % Use DejaVu Mono for monospaced font
  \usepackage[scaled=0.80]{DejaVuSansMono}
  % Some declarations for the Gantt diagram example
  \DeclareUnicodeCharacter{1D49}{\textsuperscript{e}} % "ᵉ" character
  \DeclareUnicodeCharacter{02B3}{\textsuperscript{r}} % "ʳ" character
  \DeclareUnicodeCharacter{1D52}{\textsuperscript{o}} % "ᵒ" character
\fi
\usepackage{microtype}

%%%% PAGE GEOMETRY %%%%
\usepackage{geometry}

%%%% PAGE STYLE %%%%%
\usepackage{fancyhdr}

%%%% PARAGRAPH STYLE %%%%
% Blank line between paragraphs instead of first line indenting
\usepackage{parskip}

%%%% TITLES STYLE %%%%
\usepackage{titlesec}

%%%% COLOR DEFINITIONS %%%%
\usepackage[svgnames]{xcolor}

%%%% TABLES %%%%
%\usepackage{array}
\usepackage{tabularx,booktabs,wrapfig}
\usepackage{enumitem}

%%%% ACRONYMS %%%%
\usepackage{acro}
% class `abbrev': abbreviations:
\DeclareAcronym{EU}{
  short = EU,
  long  = European Union,
  tag = abbrev
}

\DeclareAcronym{ETSETB}{
  short = ETSETB,
  long  = Escola Tècnica Superior d'Enginyeria de Telecomunicació de Barcelona,
  tag = abbrev
}

\DeclareAcronym{GREELEC}{
  short = GREELEC,
  long  = Grau en Enginyeria Electrònica de Telecomunicació,
  tag = abbrev
}

\DeclareAcronym{GEF}{
  short = GEF,
  long  = Grau en Enginyeria Física,
  tag = abbrev
}

\DeclareAcronym{VQE}{
  short = VQE,
  long  = Variational Quantum Eigensolver,
  tag = abbrev
}

\DeclareAcronym{DQS}{
  short = DQS,
  long  = Digital Quantum Simulation,
  tag = abbrev
}

\DeclareAcronym{AQS}{
  short = AQS,
  long  = Analog Quantum Simulation,
  tag = abbrev
}

\DeclareAcronym{MPS}{
  short = MPS,
  long  = Matrix Product States,
  tag = abbrev
}

\DeclareAcronym{PEPS}{
  short = PEPS,
  long  = Projected Entangled Pair States,
  tag = abbrev
}

\DeclareAcronym{FCI}{
  short = FCI,
  long  = Full Configuration Interaction,
  tag = abbrev
}

\DeclareAcronym{CCSD}{
  short = CCSD,
  long  = Coupled Cluster with Singles and Doubles,
  tag = abbrev
}

\DeclareAcronym{ADAPT-VQE}{
  short = ADAPT-VQE,
  long  = Adaptive Variational Quantum Eigensolver,
  tag = abbrev
}

\DeclareAcronym{QNG}{
  short = QNG,
  long  = Quantum Natural Gradient,
  tag = abbrev
}

\DeclareAcronym{GD}{
  short = GD,
  long  = Gradient Descent,
  tag = abbrev
}

\DeclareAcronym{NMomentum}{
  short = NMomentum,
  long  = Nesterov Momentum,
  tag = abbrev
}

\DeclareAcronym{RMSProp}{
  short = RMSProp,
  long  = Root Mean Square Propagation,
  tag = abbrev
}

\DeclareAcronym{Adagrad}{
  short = Adagrad,
  long  = Adaptive Gradient Algorithm,
  tag = abbrev
}

\DeclareAcronym{Adam}{
  short = Adam,
  long  = Adaptive Moment Estimation,
  tag = abbrev
}

\DeclareAcronym{NISQ}{
  short = NISQ,
  long  = Noisy Intermediate-Scale Quantum,
  tag = abbrev
}

\DeclareAcronym{GPU}{
  short = GPU,
  long  = Graphics Processing Unit,
  tag = abbrev
}

\DeclareAcronym{CPU}{
  short = CPU,
  long  = Central Processing Unit,
  tag = abbrev
}

\DeclareAcronym{UCCSD}{
  short = UCCSD,
  long  = Unitary Coupled Cluster with Singles and Doubles,
  tag = abbrev
}

\DeclareAcronym{HF}{
  short = HF,
  long  = Hartree-Fock,
  tag = abbrev
}



%%%% GRAPHICS %%%%
\usepackage{graphicx}
\usepackage{caption}        %
\usepackage{tikz}           % Generic vector graphics language
\usepackage{pgfgantt}       % Gantt diagrams. See "pgfgantt" manual for reference

%%%% ELECTRONIC CIRCUITS %%%%
\usepackage{circuitikz}

%%%% PLOTTING %%%%
\usepackage{pgfplots}

%%%% COLOR BOXES %%%%
\usepackage{tcolorbox}

%%%% BIBLIOGRAPHY (using biblatex) %%%%
\usepackage[section,numbib]{tocbibind}  
\usepackage[backend=biber,style=numeric]{biblatex}
\addbibresource{TFG.bib}


%%%% QUOTES %%%%
\usepackage{csquotes}

%%%% HYPERLINKS %%%%
\usepackage{hyperref}

%%%% ADDED %%%%%
\usepackage{varwidth}
\usepackage{float}
\usepackage{listings}
\usepackage{eurosym}
\usepackage{subcaption}
\usepackage{multirow}