%%%% PLEASE REPLACE ENTIRELY WITH YOUR OWN CONTENT %%%%


\chapter{Methodology / project development}

In this chapter, the methodology used in the completion of the work will be detailed. Its aim is to offer a thorough account of the approaches and techniques used, ensuring replicability and academic rigor. It will not only cover the research methods and measurement techniques employed but will also delve into the specifics of software and hardware development. Whether the project involves qualitative analysis, quantitative measurements, computational modeling, or physical prototyping, this chapter should elucidate how each component contributes to the overall objectives.

In addition to describing the methods themselves, the chapter will also provide justifications for why specific methods were chosen over others. For example, it may explain the choice of a particular programming language, statistical test, or experimental setup. The chapter will also address the limitations of the methodology and how these have been mitigated or accounted for. Readers should come away with a clear understanding of how the project's development has been carried out, why certain choices were made, and how these methods serve to fulfill the initially established objectives.


\section{Framework Selection}
To make the decision on which framework to use, we compared the documentation of the two quantum simulation frameworks available in the market: PennyLane and Qiskit. These are the most comprehensive frameworks with similar features available at the time of creating this project. After reviewing the documentation, we ultimately chose to use PennyLane for two reasons.

The first reason was the amount of documentation related to quantum simulation. Once we started looking into how others were using these resources, we realized that in the field of molecular simulation, the existing documentation—both theoretical and especially practical—was substantially greater. This provided us with more examples to begin developing our project.

The second reason for our choice was the frequent major changes implemented by Qiskit. We realized that while Qiskit is a tool that promises to be very good, it has historically undergone significant structural changes.

For these reasons, this project has been developed using the PennyLane framework. Below, we will observe how the project has been developed and explain the reasons behind the decisions made.

\section{Project Structuring}

\subsection{Code Organization}
\begin{ProjectStructure}
  \texttt{quantum\_simulation\_project/}
  \begin{itemize}[label={}, left=1em]
      \item \texttt{config/}
      \begin{itemize}[label={}, left=2em]
          \item \texttt{config\_functions.py}: Configuration functions for the project.
          \item \texttt{molecules.json}: Molecule data.
          \item \texttt{\_\_pycache\_\_}: Python cache files.
      \end{itemize}
      \item \texttt{main.py}: Main program file.
      \item \texttt{modules/}
      \begin{itemize}[label={}, left=2em]
          \item \texttt{ansatz\_preparer.py}: Quantum ansatz preparation.
          \item \texttt{hamiltonian\_builder.py}: Molecular Hamiltonian construction.
          \item \texttt{molecule\_manager.py}: Molecular data management.
          \item \texttt{opt\_mol.py}: Molecular optimization.
          \item \texttt{optimizer.py}: Optimization algorithms.
          \item \texttt{visualizer.py}: Visualization tools.
          \item \texttt{\_\_pycache\_\_}: Python cache files.
      \end{itemize}
      \item \texttt{temp\_results\_autograd/}
      \begin{itemize}[label={}, left=2em]
          \item \texttt{energy\_evolution.png}: Energy evolution graph.
          \item \texttt{filtered\_report\_autograd.txt}: Filtered results report.
          \item \texttt{final\_geometries\_3D.png}: Image of the final 3D geometries.
          \item \texttt{nuclear\_coordinates.png}: Nuclear coordinates.
          \item \texttt{output.txt}: Program data output.
          \item \texttt{profile\_output\_autograd.txt}: Autograd profile output.
      \end{itemize}
      \item \texttt{test/}: Directory for tests.
  \end{itemize}
\end{ProjectStructure}

\subsubsection{Modular Structure}
After deciding on the interface to use and implementing the first version of the code, we decided to reorganize the project to make it more precise and modular. This structure offers the possibility to easily add more lines of code and functionalities.

\subsubsection{Description of Files and Directories}

\begin{itemize}
    \item \textbf{Main File (\texttt{quantum\_simulation.py})}: Explains its function as the entry point of the program.
    \item \textbf{Auxiliary Modules (\texttt{mol\_optimizer.py})}: Details the included functions and classes, and how responsibilities are divided.
    \item \textbf{Data Files (\texttt{data/})}: Describes how and where input data and results are stored.
    \item \textbf{Dependencies (\texttt{requirements.txt})}: Mentions the importance of managing dependencies for reproducibility.
\end{itemize}


\subsection{Version Control}

Version control was managed using Git, allowing detailed tracking of changes and facilitating continuous collaboration with the supervisors on the project. Primarily, at the start of the project, a single branch was used to develop the project and explore the framework's possibilities. Once a stable version was achieved, branches were created to conduct tests and develop new functionalities. The first branches created were for the different interface versions. In each branch, the code was refined so that the same code would run across the various interfaces. Finally, only the interface changes that proved most suitable for the project were merged back into the main branch.

\section{Development and Implementation}

In our simulator, the method we use to find the energy of the ground state of a quantum system is the VQE. We have already explained the concept of VQE in the state of the art chapter; now we will explain how we have implemented it in our project and how we have integrated it.

\paragraph{Principle of VQE:}

The VQE is based on the variational principle, which states that the expected energy of any approximate state \( |\psi(\theta)\rangle \) is always greater than or equal to the real ground state energy \( E_0 \):

\[
E(\theta) = \langle \psi(\theta) | H | \psi(\theta) \rangle \geq E_0
\]

We have already discussed this concept, but it is necessary to emphasize it as it is the foundation of the entire algorithm. The idea is to find the parameters \( \theta \) that minimize the expected energy, thereby approaching the real value of the ground state energy.

\bigskip

Next, we will detail how the VQE is implemented in our project, explaining how each component has been developed.

\subsection{Implementation of VQE in Our Simulator}

The VQE algorithm in our simulator consists of several key steps, which we describe below:

\begin{enumerate}
  \item \textbf{Preparation of the Quantum Ansatz}
  
  The first step to implement the VQE algorithm (Variational Quantum Eigensolver) is to prepare the quantum state of the system. This is achieved using an \textbf{ansatz}, which is a parameterized quantum circuit designed to approximate the system's ground state. In our project, we have implemented an adaptive ansatz that selects the most relevant excitations, thus allowing a more efficient and accurate representation of the ground state.
  
  \begin{tcblisting}{colback=gray!5!white,colframe=gray!75!black,listing only,
    title=Preparation of the Quantum Ansatz, fonttitle=\bfseries, breakable, enhanced jigsaw, leftupper=8mm,
    listing options={language=Python, basicstyle=\ttfamily\small,
    showstringspaces=false, numbers=left, numberstyle=\footnotesize, stepnumber=1, numbersep=8pt}}
def prepare_ansatz(params, hf_state, selected_excitations, 
spin_orbitals):
    qml.BasisState(hf_state, wires=range(spin_orbitals))
    for i, exc in enumerate(selected_excitations):
        if len(exc) == 2:
            qml.SingleExcitation(params[i], wires=exc)
        elif len(exc) == 4:
            qml.DoubleExcitation(params[i], wires=exc)
  \end{tcblisting}
  
  This ansatz starts from the Hartree-Fock state, represented by \texttt{hf\_state}, and applies a series of single and double excitation operations, each parameterized by an angle \(\theta\). The selected excitations are applied using PennyLane's \texttt{SingleExcitation} and \texttt{DoubleExcitation} gates, allowing the construction of a quantum state that captures the system's electronic correlations.
  
  \item \textbf{Definition of the Cost Function}
  
  With the ansatz defined, the next step is to establish a cost function that evaluates the expected energy of the system given a set of parameters \(\theta\). In our implementation, this cost function is defined within \texttt{update\_parameters\_and\_coordinates} and calculates the expected value of the molecular Hamiltonian:
  
  \begin{tcblisting}{colback=gray!5!white,colframe=gray!75!black,listing only,
    title=Definition of the Cost Function, fonttitle=\bfseries, breakable, enhanced jigsaw, leftupper=8mm,
    listing options={language=Python, basicstyle=\ttfamily\small,
    showstringspaces=false, numbers=left, numberstyle=\footnotesize, stepnumber=1, numbersep=8pt}}
@qml.qnode(dev, interface=interface)
def cost_fn(params):
    prepare_ansatz(params, hf_state, selected_excitations, spin_orbitals)
    return qml.expval(hamiltonian)
  \end{tcblisting}
  
  This function is essential for evaluating \(E(\theta)\). By calculating the expected value of the Hamiltonian, we can quantify how close our approximate state is to the true ground state.
  
  \item \textbf{Optimization of the Parameters}
  
  Finally, we use classical optimizers, such as \texttt{GradientDescentOptimizer}, to adjust the parameters \(\theta\) and minimize the expected energy \(E(\theta)\). This process is critical for finding the quantum state that best approximates the system's ground state.
  
  \begin{tcblisting}{colback=gray!5!white,colframe=gray!75!black,listing only,
    title=Optimization of the Parameters, fonttitle=\bfseries, breakable, enhanced jigsaw, leftupper=8mm,
    listing options={language=Python, basicstyle=\ttfamily\small,
    showstringspaces=false, numbers=left, numberstyle=\footnotesize, stepnumber=1, numbersep=8pt}}
params, energy = opt.step_and_cost(cost_fn, params)
  \end{tcblisting}
  
  In this step, the parameters \(\theta\) are iteratively updated to reduce the value of the cost function. The \texttt{step\_and\_cost} method performs a parameter update and returns the associated energy, facilitating the tracking of the optimization progress.
\end{enumerate}

\textbf{Optimization Cycle}

In our simulation, we have utilized these basic concepts; however, since the objective of our simulation is to optimize the molecular geometry, we have had to modify the algorithm somewhat. In our case, we have implemented an optimization loop where, in each iteration, the ansatz parameters and the nuclear coordinates are optimized. Below, we will explain the steps of the loop and delve deeper into the methodology used and the steps of each part of the loop.

\begin{itemize}
    \item \textbf{Construction of the Molecular Hamiltonian:} The Hamiltonian of the system is generated for the current molecular geometry, incorporating any changes in the nuclear coordinates.
    \item \textbf{Calculation of the Operator Gradients:} The energy gradients with respect to each operator in the excitation pool are calculated, identifying which excitations contribute most to the energy reduction.
    \item \textbf{Selection of the Most Significant Operator Based on Gradients:} The operator with the largest gradient (in absolute value) is selected to be included in the ansatz, ensuring that the updates are the most effective.
    \item \textbf{Updating the Ansatz Parameters and Nuclear Coordinates:} The ansatz parameters \(\theta\) and, if necessary, the positions of the nuclei are adjusted using optimization techniques, aiming to minimize the system's total energy.
\end{itemize}

The cycle continues until the predefined convergence criteria are met or the established maximum number of iterations is completed. This iterative approach ensures that the ansatz is progressively refined, incorporating the most relevant excitations and adjusting the parameters to accurately approximate the system's ground state.

\subsection{Hamiltonian Construction Process}

\begin{enumerate}
    \item \textbf{Definition of Molecular Geometry:}
    
    The geometry is specified by the atomic symbols and the Cartesian coordinates of each atom in the molecule:
    
    \begin{tcblisting}{colback=gray!5!white,colframe=gray!75!black,listing only,
      title=Geometry Definition, fonttitle=\bfseries, breakable, enhanced jigsaw, leftupper=8mm,
      listing options={language=Python, basicstyle=\ttfamily\small,
      showstringspaces=false, numbers=left, numberstyle=\footnotesize, stepnumber=1, numbersep=8pt}}
symbols = ['H', 'H']
x_init = np.array([0.0, 0.0, 0.0, 0.0, 0.0, 0.74]) 
    \end{tcblisting}
    
    
    \item \textbf{Hamiltonian Construction:}
    
    The \texttt{build\_hamiltonian} function generates the molecular Hamiltonian using PennyLane's functions:
    
    \begin{tcblisting}{colback=gray!5!white,colframe=gray!75!black,listing only,
      title=Hamiltonian Build, fonttitle=\bfseries, breakable, enhanced jigsaw, leftupper=8mm,
      listing options={language=Python, basicstyle=\ttfamily\small,
      showstringspaces=false, numbers=left, numberstyle=\footnotesize, stepnumber=1, numbersep=8pt}}
def build_hamiltonian(x, symbols, charge=0, mult=1, basis_name='sto-3g'):
    x = np.array(x)
    coordinates = x.reshape(-1, 3)
    hamiltonian, qubits = qml.qchem.molecular_hamiltonian(
        symbols, coordinates, charge=charge, mult=mult, basis=basis_name
    )
    h_coeffs, h_ops = hamiltonian.terms()
    h_coeffs = np.array(h_coeffs)
    hamiltonian = qml.Hamiltonian(h_coeffs, h_ops)
    return hamiltonian
    \end{tcblisting}
    \textbf{Note:}
    A basis set, such as 'sto-3g', is selected, which is a predefined set of basis functions to represent atomic orbitals in a simplified manner. This makes the simulation more efficient.
    
    \textbf{Function Description:}
    
    This function generates the qubit Hamiltonian of a molecule by transforming the electronic Hamiltonian in second quantization into the Pauli matrix framework. Additionally, it allows for the incorporation of net charge effects, spin multiplicity, and an active space defined by a specific number of electrons and orbitals, optimizing the quantum simulation of molecular systems.
    
\end{enumerate}
\subsection{Parameter and Geometry Optimization}

\textbf{Has to be finished}

\textbf{To be continued...}
