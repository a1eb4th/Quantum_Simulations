%%%% PLEASE REPLACE ENTIRELY WITH YOUR OWN CONTENT %%%%


\chapter{Methodology / project development}

In this chapter, the methodology used in the completion of the work will be detailed. Its aim is to offer a thorough account of the approaches and techniques used, ensuring replicability and academic rigor. It will not only cover the research methods and measurement techniques employed but will also delve into the specifics of software and hardware development. Whether the project involves qualitative analysis, quantitative measurements, computational modeling, or physical prototyping, this chapter should elucidate how each component contributes to the overall objectives.

In addition to describing the methods themselves, the chapter will also provide justifications for why specific methods were chosen over others. For example, it may explain the choice of a particular programming language, statistical test, or experimental setup. The chapter will also address the limitations of the methodology and how these have been mitigated or accounted for. Readers should come away with a clear understanding of how the project's development has been carried out, why certain choices were made, and how these methods serve to fulfill the initially established objectives.

\section{Tools and Frameworks Selection} 
\subsection{Framework Selection}
To make the decision on which framework to use, we compared the documentation of the two quantum simulation frameworks available in the market: PennyLane and Qiskit. These are the most comprehensive frameworks with similar features available at the time of creating this project. After reviewing the documentation, we ultimately chose to use PennyLane for two reasons.

The first reason was the amount of documentation related to quantum simulation. Once we started looking into how others were using these resources, we realized that in the field of molecular simulation, the existing documentation—both theoretical and especially practical—was substantially greater. This provided us with more examples to begin developing our project.

The second reason for our choice was the frequent major changes implemented by Qiskit. We realized that while Qiskit is a tool that promises to be very good, it has historically undergone significant structural changes.

For these reasons, this project has been developed using the PennyLane framework. Below, we will observe how the project has been developed and explain the reasons behind the decisions made.

\section{Project Structuring}

After deciding on the interface to use and implementing the first version of the code, we decided to reorganize the project to make it more precise and modular. This structure offers the possibility to easily add more lines of code and functionalities.
\subsection{Code Organization}
\begin{ProjectStructure}
  \texttt{quantum\_simulation\_project/}
  \begin{itemize}[label={}, left=1em]
      \item \texttt{config/}
      \begin{itemize}[label={}, left=2em]
          \item \texttt{config\_functions.py}: Configuration functions for the project.
          \item \texttt{molecules.json}: Molecule data.
          \item \texttt{\_\_pycache\_\_}: Python cache files.
      \end{itemize}
      \item \texttt{main.py}: Main program file.
      \item \texttt{modules/}
      \begin{itemize}[label={}, left=2em]
          \item \texttt{ansatz\_preparer.py}: Quantum ansatz preparation.
          \item \texttt{hamiltonian\_builder.py}: Molecular Hamiltonian construction.
          \item \texttt{molecule\_manager.py}: Molecular data management.
          \item \texttt{opt\_mol.py}: Molecular optimization.
          \item \texttt{optimizer.py}: Optimization algorithms.
          \item \texttt{visualizer.py}: Visualization tools.
          \item \texttt{\_\_pycache\_\_}: Python cache files.
      \end{itemize}
      \item \texttt{temp\_results\_autograd/}
      \begin{itemize}[label={}, left=2em]
          \item \texttt{energy\_evolution.png}: Energy evolution graph.
          \item \texttt{filtered\_report\_autograd.txt}: Filtered results report.
          \item \texttt{final\_geometries\_3D.png}: Image of the final 3D geometries.
          \item \texttt{nuclear\_coordinates.png}: Nuclear coordinates.
          \item \texttt{output.txt}: Program data output.
          \item \texttt{profile\_output\_autograd.txt}: Autograd profile output.
      \end{itemize}
      \item \texttt{test/}: Directory for tests.
  \end{itemize}
\end{ProjectStructure}

This file serves as the entry point of the program. It initializes the initial conditions, sets up the molecule's configuration, defines optimization options, and orchestrates the execution of the quantum simulation and geometry optimization process. It is the first file to execute when running the application.

\textbf{Auxiliary Modules (\texttt{modules/})}:  
This directory contains various modules that encapsulate the core logic and operations of the project:
\begin{itemize}
    \item \texttt{ansatz\_preparer.py}: Includes functions for constructing the quantum circuit from the variational ansatz, preparing states, and applying excitations.
    \item \texttt{hamiltonian\_builder.py}: Responsible for generating the molecular Hamiltonian based on nuclear coordinates and the selected electronic basis.
    \item \texttt{molecule\_manager.py}: Manages information related to the molecule, such as its charge, multiplicity, number of electrons, and required orbitals.
    \item \texttt{opt\_mol.py}: Functions that orchestrate the molecular optimization process, invoking the optimizer and recording the simulation's progress.
    \item \texttt{optimizer.py}: Implements the optimization logic, combining routines for evaluating the cost (energy) with the selected optimization algorithms, updating parameters, and geometries.
    \item \texttt{visualizer.py}: Generates graphical outputs and reports that display the evolution of energy, nuclear coordinates, and other relevant data during optimization.
\end{itemize}

\textbf{Temporary Results Directories (\texttt{temp\_results\_autograd})}:  
    Several directories with names starting with \texttt{temp\_results\_autograd} store output data, time logs, and visualizations generated for different simulations and configurations. For documentation purposes, these directories are treated as a single repository for temporary results.

\textbf{Dependencies (\texttt{requirements.txt})}:  
    This file lists the required libraries and their versions to reproduce the project's execution environment. Keeping it updated ensures reproducibility of the simulation across different systems, facilitating the installation of necessary dependencies.


\subsection{Version Control}

Version control was managed using Git, allowing detailed tracking of changes and facilitating continuous collaboration with the supervisors on the project. Primarily, at the start of the project, a single branch was used to develop the project and explore the framework's possibilities. Once a stable version was achieved, branches were created to conduct tests and develop new functionalities. The first branches created were for the different interface versions. In each branch, the code was refined so that the same code would run across the various interfaces. Finally, only the interface changes that proved most suitable for the project were merged back into the main branch.

\section{Development and Implementation}

The \textit{Variational Quantum Eigensolver} (VQE) was chosen as the primary method to estimate the ground state energy of the studied quantum system. VQE combines limited quantum processing (measurements and applicability in moderately deep circuits) with classical optimization techniques. Its selection is justified by:

\begin{itemize}
    \item \textbf{Suitability for NISQ devices:} VQE is particularly well-suited for noisy intermediate-scale quantum (NISQ) devices, as it requires circuits of relatively low depth.
    \item \textbf{Flexible Ansatz:} It allows the use of various adaptive variational ansätze that capture essential electronic correlations.
    \item \textbf{Direct coupling to classical optimizers:} The VQE cost function (the expected energy) can be minimized with a wide range of classical methods, making it easy to experiment with different optimizers.
\end{itemize}

The core principle of VQE is the variational theorem, which guarantees that the expected energy of the ansatz is always an upper bound to the true ground state energy. By optimizing the ansatz parameters, the algorithm progressively approaches the actual energy minimum. We have already explained the concept of VQE in the state of the art chapter; now we will explain how we have implemented it in our project and how we have integrated it.

\paragraph{Principle of VQE:}

The VQE is based on the variational principle, which states that the expected energy of any approximate state \( |\psi(\theta)\rangle \) is always greater than or equal to the real ground state energy \( E_0 \):

\[
E(\theta) = \langle \psi(\theta) | H | \psi(\theta) \rangle \geq E_0
\]

We have already discussed this concept, but it is necessary to emphasize it as it is the foundation of the entire algorithm. The idea is to find the parameters \( \theta \) that minimize the expected energy, thereby approaching the real value of the ground state energy.

\bigskip

Next, we will detail how the VQE is implemented in our project, explaining how each component has been developed.
\subsection{Hamiltonian Construction Process}

\begin{enumerate}
    \item \textbf{Definition of Molecular Geometry:}
    
    The geometry is specified by the atomic symbols and the Cartesian coordinates of each atom in the molecule:
    
    \begin{tcblisting}{colback=gray!5!white,colframe=gray!75!black,listing only,
      title=Geometry Definition, fonttitle=\bfseries, breakable, enhanced jigsaw, leftupper=8mm,
      listing options={language=Python, basicstyle=\ttfamily\small,
      showstringspaces=false, numbers=left, numberstyle=\footnotesize, stepnumber=1, numbersep=8pt}}
symbols = ['H', 'H']
x_init = np.array([0.0, 0.0, 0.0, 0.0, 0.0, 0.74]) 
    \end{tcblisting}
    
    
    \item \textbf{Hamiltonian Construction:}
    
    The \texttt{build\_hamiltonian} function generates the molecular Hamiltonian using PennyLane's functions:
    
    \begin{tcblisting}{colback=gray!5!white,colframe=gray!75!black,listing only,
      title=Hamiltonian Build, fonttitle=\bfseries, breakable, enhanced jigsaw, leftupper=8mm,
      listing options={language=Python, basicstyle=\ttfamily\small,
      showstringspaces=false, numbers=left, numberstyle=\footnotesize, stepnumber=1, numbersep=8pt}}
def build_hamiltonian(x, symbols, charge=0, mult=1, basis_name='sto-3g'):
    x = np.array(x)
    coordinates = x.reshape(-1, 3)
    hamiltonian, qubits = qml.qchem.molecular_hamiltonian(
        symbols, coordinates, charge=charge, mult=mult, basis=basis_name
    )
    h_coeffs, h_ops = hamiltonian.terms()
    h_coeffs = np.array(h_coeffs)
    hamiltonian = qml.Hamiltonian(h_coeffs, h_ops)
    return hamiltonian
    \end{tcblisting}
    \textbf{Note:}
    A basis set, such as 'sto-3g', is selected, which is a predefined set of basis functions to represent atomic orbitals in a simplified manner. This makes the simulation more efficient.
    
    \textbf{Function Description:}
    
    This function generates the qubit Hamiltonian of a molecule by transforming the electronic Hamiltonian in second quantization into the Pauli matrix framework. Additionally, it allows for the incorporation of net charge effects, spin multiplicity, and an active space defined by a specific number of electrons and orbitals, optimizing the quantum simulation of molecular systems.
    
\end{enumerate}

\subsection{Adaptive Ansatz Construction}
The preparation of the quantum state is a critical step in implementing the Variational Quantum Eigensolver (VQE) algorithm. This is accomplished through an \textbf{ansatz}, a parameterized quantum circuit designed to approximate the ground state of the system. In our simulator, we have developed an adaptive ansatz that dynamically selects the most relevant excitations based on their contribution to reducing the system's energy. This approach enhances both the efficiency and accuracy of the ground state representation by constructing the circuit iteratively, focusing only on the most impactful excitations. 

Additionally, we have implemented a non-adaptive ansatz for comparison purposes. This fixed ansatz applies a predefined set of excitations and serves as a reference to highlight the advantages of the adaptive approach (UCSSD), particularly in improving the efficiency and precision of the simulation.

A continuación se muestra el código que permite la construcción del ansatz adaptativo:
\begin{tcblisting}{colback=gray!5!white,colframe=gray!75!black,listing only,
    title= UCSSD ansatz, fonttitle=\bfseries, breakable, enhanced jigsaw, leftupper=8mm,
    listing options={language=Python, basicstyle=\ttfamily\small,
    showstringspaces=false, numbers=left, numberstyle=\footnotesize, stepnumber=1, numbersep=8pt}}
def prepare_ansatz_uccsd(params, hf_state, selected_excitations, spin_orbitals):
qml.BasisState(hf_state, wires=range(spin_orbitals))
for i, exc in enumerate(selected_excitations):
    if len(exc) == 2:
        qml.SingleExcitation(params[i], wires=exc)
    elif len(exc) == 4:
        qml.DoubleExcitation(params[i], wires=exc)
\end{tcblisting}

The adaptive ansatz iteratively builds the quantum circuit by selecting the most relevant excitations from the molecular system. It starts from the Hartree-Fock state, where parameterized excitations are applied to capture electronic correlations. At each iteration, the energy gradients of excitations from a predefined pool are calculated, and the excitation that most significantly reduces the energy is selected. This excitation is then incorporated into the circuit, expanding the set of variational parameters. The process is repeated until a convergence criterion is met, progressively refining the representation of the ground state.

The adaptive method offers significant advantages over traditional approaches by dynamically constructing the ansatz and incorporating only the excitations that have the greatest impact on energy reduction. This avoids the inclusion of irrelevant operators, reducing the circuit complexity and the number of parameters to optimize. Furthermore, it improves computational efficiency by focusing on the most significant correlations within the system, accelerating convergence to the ground state. In contrast, traditional methods use a fixed ansatz that does not adapt to the molecular system’s characteristics, leading to larger, less efficient circuits with higher resource demands and no guarantee of improved accuracy.

\subsection{Cost Function Definition}
With the ansatz defined, the next step is to establish a cost function that evaluates the expected energy of the system given a set of parameters \(\theta\). In our implementation, this cost function is defined within \texttt{update\_parameters\_and\_coordinates} and calculates the expected value of the molecular Hamiltonian:
  
  \begin{tcblisting}{colback=gray!5!white,colframe=gray!75!black,listing only,
    title=Definition of the Cost Function, fonttitle=\bfseries, breakable, enhanced jigsaw, leftupper=8mm,
    listing options={language=Python, basicstyle=\ttfamily\small,
    showstringspaces=false, numbers=left, numberstyle=\footnotesize, stepnumber=1, numbersep=8pt}}
@qml.qnode(dev, interface=interface)
def cost_fn(params):
    prepare_ansatz(params, hf_state, selected_excitations, spin_orbitals)
    return qml.expval(hamiltonian)
  \end{tcblisting}
  
This function is essential for evaluating \(E(\theta)\). By calculating the expected value of the Hamiltonian, we can quantify how close our approximate state is to the true ground state.

\subsection{Optimization Process}

The optimization process in our simulator integrates the refinement of both variational parameters \(\theta\) and nuclear coordinates \(\mathbf{X}\), aiming to find the global minimum of the molecular system's total energy. This is achieved through an iterative cycle that combines electronic and nuclear optimization, ensuring that the molecule reaches its equilibrium geometry and ground state energy simultaneously.

\paragraph{Steps in the Optimization Loop:}
\begin{enumerate}
    \item \textbf{Initialization:}
    
    The process begins with the initialization of the molecular geometry, nuclear coordinates, and electronic parameters. These are defined as:
    
    \begin{tcblisting}{colback=gray!5!white,colframe=gray!75!black,listing only,
      title=Initialization Example, fonttitle=\bfseries, breakable, enhanced jigsaw, leftupper=8mm,
      listing options={language=Python, basicstyle=\ttfamily\small,
      showstringspaces=false, numbers=left, numberstyle=\footnotesize, stepnumber=1, numbersep=8pt}}
symbols = ['H', 'H']
x_init = np.array([0.0, 0.0, 0.0, 0.0, 0.0, 0.74])
params = np.array([], requires_grad=True)
    \end{tcblisting}

    \item \textbf{Molecular Hamiltonian Construction:}
    
    The molecular Hamiltonian is recalculated at each step using the \texttt{build\_hamiltonian} function, incorporating updated nuclear coordinates.

    \item \textbf{Gradient Calculation and Operator Selection:}
    
    Energy gradients are computed for all excitations in the operator pool. The operator with the highest gradient is selected for inclusion in the ansatz, ensuring effective energy reduction. This step uses:
    \begin{tcblisting}{colback=gray!5!white,colframe=gray!75!black,listing only,
      title=Operator Selection, fonttitle=\bfseries, breakable, enhanced jigsaw, leftupper=8mm,
      listing options={language=Python, basicstyle=\ttfamily\small,
      showstringspaces=false, numbers=left, numberstyle=\footnotesize, stepnumber=1, numbersep=8pt}}
selected_gate, max_grad_value = select_operator(gradients, operator_pool, convergence)
if selected_gate:
    selected_excitations.append(selected_gate)
    params = np.append(params, 0.0)
    \end{tcblisting}
    
    \item \textbf{Parameter and Nuclear Coordinate Update:}
    
    Both variational parameters and nuclear positions are updated iteratively. The optimization routine leverages numerical gradients to adjust the coordinates:
    \begin{tcblisting}{colback=gray!5!white,colframe=gray!75!black,listing only,
      title=Parameter and Coordinate Update, fonttitle=\bfseries, breakable, enhanced jigsaw, leftupper=8mm,
      listing options={language=Python, basicstyle=\ttfamily\small,
      showstringspaces=false, numbers=left, numberstyle=\footnotesize, stepnumber=1, numbersep=8pt}}
params, x, energy_history, x_history, opt_state = update_parameters_and_coordinates(
    opt, opt_state, cost_fn, params, x, symbols, selected_excitations, dev, hf_state, spin_orbitals,
    learning_rate_x, convergence, interface, charge, mult, basis_name
)
    \end{tcblisting}

    \item \textbf{Visualization and Convergence:}
    
    The optimization progress is visualized with plots of energy versus iteration and 3D representations of the molecular geometry. The process stops when the predefined convergence criteria, such as energy difference below \(10^{-8}\), are met.
\end{enumerate}

\paragraph{Justification for the Combined Optimization Approach:}
Coupling the optimization of electronic parameters and nuclear geometry mirrors the interdependence between the electronic state and molecular structure. This method ensures simultaneous refinement of the molecular system, reducing computational overhead compared to sequential optimization methods. By selecting the most impactful excitations and dynamically adjusting both \(\theta\) and \(\mathbf{X}\), the algorithm achieves a high level of precision and efficiency in finding the ground state energy and equilibrium configuration.
\subsection{Visualization Tools}

Results visualization allows analyzing the progress and outcomes of quantum simulations through graphs that depict energy evolution, molecular geometries, and execution times. The tools utilized are described below:

\begin{itemize}
    \item \textbf{Energy Evolution:} Energy graphs are generated as a function of iterations. These include:
    \begin{itemize}
        \item \textit{Linear Scale:} Useful for observing large energy changes during the initial iterations.
        \item \textit{Logarithmic Scale with Offset:} Highlights small variations in the final stages of the process.
    \end{itemize}

    \item \textbf{Interatomic Distances:} Changes in the distances between atomic pairs are represented across iterations, showing how the molecular structure evolves.

    \item \textbf{3D Molecular Geometries:} Final molecular geometries are visualized using 3D plots, highlighting atomic positions and differences between configurations and optimizers.

    \item \textbf{Energy vs Time:} Graphs show energy evolution as a function of cumulative time, allowing the evaluation of each optimizer's efficiency.

    \item \textbf{Execution Times:} Execution times for each function and optimizer are stored in a CSV file for detailed efficiency analysis.
\end{itemize}

The visualization tools provide a clear understanding of the simulations' behavior, facilitating the analysis of results and the comparison of configurations.

\section{Limitations and Mitigation Measures}

Among the limitations of this approach are:
\begin{itemize}
    \item \textbf{Scalability:} As the system grows in the number of electrons and orbitals, the complexity of Hamiltonian construction and the excitation space increases exponentially.
    \item \textbf{Quantum Noise and Errors:} On real devices, noise affects measurement fidelity. Our work, primarily simulation-oriented, plans to integrate mitigation techniques in future studies.
    \item \textbf{Ansatz Choice:} Although the adaptive ansatz helps, there is no guarantee that the excitation selection is optimal. Future work might explore more complex heuristics.
\end{itemize}

To mitigate these issues, we opted for reduced basis sets, strategies such as re-initializing the optimizer when increasing the parameter space, and verifying convergence through multiple criteria (energetic and geometric).
